\documentclass{beamer}
\usepackage{catchfilebetweentags}
\usepackage{multicol}
\usepackage{tikz}
\usepackage{pdfpages}
\usetikzlibrary{matrix,shapes,arrows,positioning,chains,calc,decorations.markings,fit}
\usetheme{Singapore}

\mode<presentation>
\title{Design Considerations for the Characterization of Capacitors}
\author{Michael L. DeLibero}
\date{August, 2015}
\institute[CWRU]{Case Western Reserve University}
\bibliographystyle{plain}

\newcommand{\loadeq}[1]{\ExecuteMetaData[../equations/equations.tex]{eq#1}}

% This file contains all of the if defines for the project

% Set to true if generating a powerpoint / beamer presentation.
\newif\ifisPPT
\isPPTfalse

% Set to true if you only want the first half of the /figures/circEx/flowchart.tex image.
\newif\ifOnlyFirst
\OnlyFirstfalse

% Set to true if you only want the second half of the /figures/circEx/flowchart.tex image.
\newif\ifOnlySecond
\OnlySecondfalse



\isPPTtrue

\begin{document}

\begin{frame}
    \titlepage
\end{frame}

I would like to thank my thesis advisor, Dr. Merat, for his support and instruction in helping me complete this work. I would also like to thank Dr. Welsch and ARPA-E, who's leadership and funding under ARPA-E contract DE--AR000016 made this work possible. Furthermore, I would like to thank Steven Ehret, who's initial work on this contract enabled my own.



\begin{frame}
  \frametitle{Outline}
  \tableofcontents
%  \tableofcontents[pausesections]
\end{frame}

\section{Background}
\begin{frame}
    \frametitle{Background}
    \begin{itemize}
        \item In 2011, ARPA-E granted Dr. Welsch an award to develope Titanimum-based capacitors.
        \item They promise to replace tantalum.
        \item Greater energy and power density.
    \end{itemize}
\end{frame}

\section{Aim of Work}
\begin{frame}
    \frametitle{Aim of Work}
    \begin{itemize}
        \item Design instrumentation capable of testing a large number of titanium capacitors.
        \item Provide a numerical result to be able to characterize physical parameters in the capacitors.
    \end{itemize}
\end{frame}

\section {Measurement Circuitry}
This section describes the schematic design used to meet the goals set out in the Abstract (Section: \ref{sec:abstract}). The schematic can be found in the Appendix section: \ref{sec:schematic}.

\nocite{my_ieeePaper}

\subsection {DC Bias}
\label{sec:dcBias}

The DC bias signal is externally generated by a Stanford Research Supply SRS-PS350 power supply. Some of the models of this supply provide a serial input which can be used to set their output voltage. This scenario is preferred and the communications for it are handeled through the RS232 transceiver found on Schematic Page: 3.

\begin{figure}
\includegraphics[keepaspectratio=true,width=6in]{./figures/board/srsCtrl.jpg}
\centering
\caption{SRS Control Circuit}
\label{srsCtrl}
\end{figure}


Other models of this supply only provide an analog input option to control their output voltage. The specifications for this method can be seen in Table: \ref{table:srsCtrl_table}. The control signal is generated by using the microcontroller to set the DAC output over a SPI bus. This signal is then buffered and clamped to under 1.5V. Since the SRS-PS350 has a gain of 500, this clamps its output to 750VDC. The circuit is only designed to operate at up to 500VDC, but this value is still within the specified ratings of the components in the signal path.

The DAC has a 2.5V internal reference with a 14 bit resolution, giving the control signal an ideal resolution of $150\mu V$ and the SRS-PS350 an ideal output resolution of $76mV$. The update rate of this device is not important to this application because the DAC will be set infrequently and only when the system is not actively collecting data.

If different supply characteristics are needed, the SRS-PS340 can be swapped out with a different supply, but the control logic may not be supported.


\begin{table}[ht!]
\centering
\begin{tabular}{|l|l|l|}
\hline
Value & Range & Current   \\ \hline
5     & Hi    & 1A  - 1mA \\ \hline
5k    & Med   & 1mA - 1uA \\ \hline
5M    & Low   & 1uA - 1nA \\ \hline
\end{tabular}
\caption{Current Measurement Ranges}
\label{table:imeas_table}
\end{table}

\subsection{Charging Circuitry}
\label{sec:charging}

This section describes the DUT charging circuitry seen in Schematic Page: 4. This part of the circuitry is meant to only be used to prepare the DUT for a test. The current is purposely limited in order to minimize the load on the SRS-PS350. The charging operation functions by enabling the high-current measurement and charging relays, and then slowly ramping the DC bias voltage until the DUT is fully charged.


\subsection{Discharging Circuitry}

The discharging \hyperlink{sch:discharging}{circuitry} provides a means to determine the bulk capacitance value of the \gls{dut} from the RC time constant. Once the \gls{dut} is charged via the operation described in Section: \ref{sec:charging}, the charging relay will open and the high-current discharge relay will close. The transimpedance amplifier described in Section: \ref{sec:iMeas} will measure the current through the \gls{dut} as it decays.

There are 3 switched discharge stages that allow for the decay rate to be regulated. Each stage allows for a range of voltages and capacitances needed to stay within safety and operational constraints. As seen in Figure: \ref{fig:opArea}, there are two constraints on both the voltage and capacitance. The maximum voltage is constrained by either the power rating on the lowest valued resistor or the overall safety limit of $500$VDC. The minimum capacitance value is constrained by setting a $100$ \gls{adc} samples per time constant with a sampling rate of $240$KSPS.

\begin{figure}
\includegraphics[keepaspectratio=true,width=3in]{./figures/circEx/operatingArea.jpg}
\centering
%    \cite{capSite_df_vs_temp}
\caption{Operating Area}
\label{opArea}
\end{figure}



\subsection{Magnitude}
\label{sec:mag}

The magnitude \hyperlink{sch:imph}{circuitry} measures both the magnitude of the input sine wave and the magnitude of the resultant current waveform through the \gls{dut}. The design for this circuit is based on Analog Device's application note for the AD8277 \cite{absCircuit}.

The function of this circuit is to perform a full-wave rectification operation on the input signal and then pass the output to be filtered and then sampled. The first difference amplifier is configured as a voltage follower. Since it is only powered from a single rail, it passes the positive half of the waveform and shunts the negative half to ground. The second difference amplifier operates in different modes, based upon the output of the first differential amplifier. During the negative going portions of the waveform, its positive terminal is held at ground, and it operates as a unity gain, inverting amplifier. This rectifies the input signal to the output. During the positive going portions of the waveform, positive input terminal is held at the value of the input terminal. This forces the second difference amplifier to act as a voltage follower. The resultant waveform is a full-wave rectified version of the input signal. It is then bypassed with an output capacitor and fed into the \gls{adc} to calculate the value of the rms voltage.


\subsection{Phase}

The phase \hyperlink{sch:imph}{circuitry} measures both the phase of the input sine wave and the phase of the resultant current waveform through the \gls{dut}. It uses two, redundant phase measurement techniques.

The first technique utilizes the AD8302, a \gls{rf} phase detector, in a low frequency configuration as per AN-691 \cite{AD8302_AppNote}. The capacitors on the IC inputs tune the front end high pass filter cutoff to roughly 20Hz. The phase measurement is accomplished internally by a magnitude independent phase multiplier configuration. The output conisists of a 10mV per degree signal that is lowpassed at 2Hz to prevent aliasing. The filtered signal is then fed to the ADC for digitization.

The second technique is added for a low frequency comparison against the \gls{rf} phase detector. The comparator acts as a sine to square wave converter. The default configuration relies on the comparator's internal hysteresis, but provisions for wider thresholds are available if needed. The outputs of the two converters are fed into timer inputs on the microcontroller. The phase is then calculated by the \gls{mcu} by measuring the period of the signals and the time difference between their rising edges.


\subsection{Communications}

The communications \hyperlink{sch:com}{circuitry} is used to communicate to off-board processors.

\subsubsection{USB}
The \gls{usb} section is used to communicate with a PC for data logging and post processing. This circuitry centers around a FTDI FT232H serial to \gls{usb} chip. It allows seamless \gls{usb} communication to a PC's COM port with the \gls{mcu} only using a single UART port. This significantly lowers the complexity of the communication bus, as the \gls{mcu} is not responsible for running the \gls{usb} stack.

\subsubsection{RS-232}
The board also has two, bidirectional RS-232 ports. They are able to be used for general purposes, but will most often be used for communicating with a sine-wave generator and the DC Bias supply.



\renewcommand{\loadeq}[1]{%
   \ExecuteMetaData[../equations/regression.tex]{eq#1}%
}

\begin{frame}
\begin{figure}[ht!]
\ifisPPT
\noindent\makebox[\textwidth]{\includegraphics[keepaspectratio=true,width=\paperwidth]{../figures/regression/basicLSE.jpg}}
\else
\includegraphics[keepaspectratio=true,width=6in]{./figures/regression/basicLSE.jpg}
\fi
\centering
\caption{Basic LSE}
\label{fig:basicLSE}
\end{figure}


\end{frame}

\begin{frame}
\frametitle{Basic Regression}
\loadeq{ExLinEqu}
\loadeq{LSEE2}
\loadeq{LSEE2b}
\loadeq{LSEPD}
\end{frame}

\begin{frame}
\begin{figure}[h]
\includegraphics[keepaspectratio=true,width=6in]{./figures/modeling/exCapData.jpg}
\centering
\caption{GRM31MR71H105KA88 Capacitor Data}
\label{fig:exCapData}
\end{figure}

\end{frame}

\begin{frame}
\loadeq{levyGs}
\loadeq{LevyL}
\loadeq{LevyS}
\loadeq{LevyT}
\loadeq{LevyU}
\end{frame}

\begin{frame}
\loadeq{LevyAns}
\loadeq{fullModelM}
\end{frame}

\begin{frame}
\loadeq{LevyNC}
\end{frame}


\section {Conclusion}

In conclusion, the circuitry devolped in this thesis, along with the regression analysis developed, shows a promising ability to accurately model capacitors. The method shown provides the ability to model a capacitor over a 0->500VDC bias and a 100Hz->100kHz range. The regression analysis provides a means to compare new capacitors with unknown characteristics. The accuracy of the model is mostly determined by the number of parameters and typically has the greatest relative error near resonance.


\section {Future Work}
\label{sec:futureWork}

This section will describe the future work that needs to be accomplished in order to complete and further the stated goals of this thesis. It will focus on the practical circuitry implementation and other aspects needed to make it a viable tool.

\begin{itemize}
    \item Build the circuit and validate against the stated capabilities and accuracy.
    \item Increase the available frequency range for the measurement circuitry.
    \item Add additional fail-safe protection to the circuit.
    \item Update the regression technique for better accuracy (See Section: \ref{app:adtl_regression}).
    \item Evaluate the accuracy of the six term model described in Section: \ref{sec:regression} with empirical data.
\end{itemize}




\bibliography{../metadata/thesis,../figures/thesis_figs}

\end{document}

