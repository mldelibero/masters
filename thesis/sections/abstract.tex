\label{sec:abstract}
Various industry needs, such as electric vehicles and difribrullators, are requiring capacitors with a high energy and power density that can operate in the prescence of a DC bias in the 100s of volts. Researchers have been developing technologies such as \gls{ec} for lower voltages and Titanium based capacitors for the higher voltage applications. A critical step in this developement is the characterization of the capacitors operating with a high DC bias.
This thesis suggests a theoretical approach for characterizing capacitors at up to a 500VDC bias. A circuit was designed to allow a user to characterize a capacitor through discharge curves and a swept frequency analysis of the impedance. It has a frequency range of 100Hz $\rightarrow$ 100kHz, a DC bias range of 0$\rightarrow$500V, and an AC signal amplitude range of 0$\rightarrow$5V.
An iterative regression analysis was also developed to fit the empirical data to a model. An analysis of a fit of a 6 term model to a large ($>30$) term model was completed. The accuracy of the model is mostly determined by the number of parameters and typically has the greatest relative error near resonance. Outside of resonance and low frequencies, the accuracy of the fit was less than $2 \Omega$ and $2^{\circ}$. The relative error, outside of a single peak in the phase plot, stays well below $5\%$.

