\label{sec:abstract}
Various industry needs, such as electric vehicles and defibrillators, are requiring capacitors with a high energy and power density that can operate in the presence of a DC bias in the 100s of volts. Researchers have been developing technologies such as electrochemical capacitors for lower voltages and Titanium based capacitors for the higher voltage applications. A critical step in this development is the characterization of the capacitors operating with a high DC bias.
This thesis outlines a practical method to characterize a capacitor of up to tens of microfards with a DC bias up to 500 volts. A circuit has been designed to characterize a capacitor through discharge curves and a swept frequency analysis of the impedance. It has a frequency range of 100Hz $\rightarrow$ 100kHz, a DC bias range of 0$\rightarrow$500V, and an AC signal amplitude range of 0$\rightarrow$5V.
An iterative regression analysis has been developed to fit measured data to a multi-term capacitor model. Simulated data has been generated using Murata’s 34-term model for a ceramic capacitor and fit to a six-term model which incorporates dielectric absorption, leakage and resistance. The accuracy of the fit to the data is largely determined by the number of parameters in the capacitor model and typically has the greatest relative error near resonance. The six-term model fit the simulated data within $2 \Omega$ for the magnitude of the impedance and within $2^{\circ}$ for the phase of the impedance. The relative error, outside of a single peak in the phase plot, stays well below $5\%$.
