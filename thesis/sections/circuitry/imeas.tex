\subsection{Current Measurement}

The centerpiece of this part of the circuit is the transimpedance amplifier. It works by creating a virtual ground on the negative potential side of the DUT and then returning the current through a resistor to ground. This functionality is specifically useful in this application because it effectively uses the DUT to buffer the measurement circuitry from the high voltage. The transimpedance amplifier has 3 switched feedback stages that allow it to measure 9 decades of current. The output will be buffered and then filtered and digitized dependent of the specific test of interest.

The output voltage is calculated by Equation: \eqref{trans}.

\begin{equation}
\label{trans}
Vo = I*R_f
\end{equation}

The three measurement sections allow for a high precision of measurement over a wide range of current. They are listed in Table: \ref{imeas_table}.

\subsection{Current Measurement}

Each of the tests utilize the low-side current measurement circuit. The AC measurement tests provide a known voltage across the DUT and then the current is used to construct the impedance. The discharge tests use the current measurement circuit to capture the current through the DUT over time as its energy drains. 

\subsection{Current Measurement}

Each of the tests utilize the low-side current measurement circuit. The AC measurement tests provide a known voltage across the DUT and then the current is used to construct the impedance. The discharge tests use the current measurement circuit to capture the current through the DUT over time as its energy drains. 

\subsection{Current Measurement}

Each of the tests utilize the low-side current measurement circuit. The AC measurement tests provide a known voltage across the DUT and then the current is used to construct the impedance. The discharge tests use the current measurement circuit to capture the current through the DUT over time as its energy drains. 

\input{tables/imeas}

The current measurement circuit utilizes a transimpedance amplifier designed from the reference circuit in \cite{steve_thesis}. It utilizes 3 switched, feedback stages to measure 9 decades of current (See Table: \ref{imeas_table}). The circuit creates a virtual ground at the negative terminal of the DUT and then uses one of the feedback paths to create a voltage for a digitizer. Using a low-side current measurement circuit is beneficial in this circumstance because it allows low voltage circuitry to condition and measure a signal without needing to see the high DC voltage on the positive terminal of the DUT. The output voltage is calculated by Equation: \eqref{trans}.

\begin{equation}
\label{trans}
Vo = I*R_f
\end{equation}

*Add transistor in increase the Op Amp's current drive capability.
*Change capacitors in feedback loop.
*Add Sallen-key filter to output
*Explain difference in two buffering stages


The output will be buffered and then filtered and digitized dependent of the specific test of interest.



The current measurement circuit utilizes a transimpedance amplifier designed from the reference circuit in \cite{steve_thesis}. It utilizes 3 switched, feedback stages to measure 9 decades of current (See Table: \ref{imeas_table}). The circuit creates a virtual ground at the negative terminal of the DUT and then uses one of the feedback paths to create a voltage for a digitizer. Using a low-side current measurement circuit is beneficial in this circumstance because it allows low voltage circuitry to condition and measure a signal without needing to see the high DC voltage on the positive terminal of the DUT. The output voltage is calculated by Equation: \eqref{trans}.

\begin{equation}
\label{trans}
Vo = I*R_f
\end{equation}

*Add transistor in increase the Op Amp's current drive capability.
*Change capacitors in feedback loop.
*Add Sallen-key filter to output
*Explain difference in two buffering stages


The output will be buffered and then filtered and digitized dependent of the specific test of interest.



The current measurement circuit utilizes a transimpedance amplifier designed from the reference circuit in \cite{steve_thesis}. It utilizes 3 switched, feedback stages to measure 9 decades of current (See Table: \ref{imeas_table}). The circuit creates a virtual ground at the negative terminal of the DUT and then uses one of the feedback paths to create a voltage for a digitizer. Using a low-side current measurement circuit is beneficial in this circumstance because it allows low voltage circuitry to condition and measure a signal without needing to see the high DC voltage on the positive terminal of the DUT. The output voltage is calculated by Equation: \eqref{trans}.

\begin{equation}
\label{trans}
Vo = I*R_f
\end{equation}

*Add transistor in increase the Op Amp's current drive capability.
*Change capacitors in feedback loop.
*Add Sallen-key filter to output
*Explain difference in two buffering stages


The output will be buffered and then filtered and digitized dependent of the specific test of interest.



