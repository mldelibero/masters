\subsection{Current Measurement}
\label{sec:iMeas}

This section describes the current measurement circuitry seen on Schematic Page: 5). In light of the high DC bias voltage present in the system (Section: \ref{sec:dcBias}) this circuitry utilizes a low-side current measurement technique. This allows the measurement electronics to operate at relative low (5VDC) voltages while measuring current from the DUT with a DC voltage in the 100s of volts.

\begin{table}[ht!]
\centering
\begin{tabular}{|l|l|l|}
\hline
Value & Range & Current   \\ \hline
5     & Hi    & 1A  - 1mA \\ \hline
5k    & Med   & 1mA - 1uA \\ \hline
5M    & Low   & 1uA - 1nA \\ \hline
\end{tabular}
\caption{Current Measurement Ranges}
\label{table:imeas_table}
\end{table}


The current measurement circuit utilizes a transimpedance amplifier designed from the reference circuit in \cite{steve_thesis}. It utilizes 3 switched, feedback stages to measure 9 decades of current (See Table: \ref{table:imeas_table}). The circuit operates by creating a virtual ground at the negative terminal of the DUT, and then directing the current through one of its feedback paths. The resultant voltage (Equation: \eqref{equ:trans}) is then filtered and sent to the next stage for further signal conditioning.

\begin{equation}
\label{equ:trans}
Vo = I*R_f
\end{equation}

The AD817 op-amp has the precision and bandwidth needed to operate in the transimpedance amplifier, but not the drive strength for th high current range. This limitation is overcome by inserting a BJT based current booster in the feedback path. This preserves the benefits of the op-amp, while providing the ability to drive a high-current signal.

