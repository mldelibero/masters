\subsection{Current Measurement}
\label{sec:iMeas}

In light of the high DC bias voltage present in the system, the current measurement \hyperlink{sch:discharging}{circuitry} utilizes a low-side current measurement technique. This allows the measurement electronics to operate at relative low (5VDC) voltages while measuring current from the \gls{dut} with a DC voltage in the 100s of volts.

\subsection{Current Measurement}

Each of the tests utilize the low-side current measurement circuit. The AC measurement tests provide a known voltage across the DUT and then the current is used to construct the impedance. The discharge tests use the current measurement circuit to capture the current through the DUT over time as its energy drains. 

\subsection{Current Measurement}

Each of the tests utilize the low-side current measurement circuit. The AC measurement tests provide a known voltage across the DUT and then the current is used to construct the impedance. The discharge tests use the current measurement circuit to capture the current through the DUT over time as its energy drains. 

\subsection{Current Measurement}

Each of the tests utilize the low-side current measurement circuit. The AC measurement tests provide a known voltage across the DUT and then the current is used to construct the impedance. The discharge tests use the current measurement circuit to capture the current through the DUT over time as its energy drains. 

\input{tables/imeas}

The current measurement circuit utilizes a transimpedance amplifier designed from the reference circuit in \cite{steve_thesis}. It utilizes 3 switched, feedback stages to measure 9 decades of current (See Table: \ref{imeas_table}). The circuit creates a virtual ground at the negative terminal of the DUT and then uses one of the feedback paths to create a voltage for a digitizer. Using a low-side current measurement circuit is beneficial in this circumstance because it allows low voltage circuitry to condition and measure a signal without needing to see the high DC voltage on the positive terminal of the DUT. The output voltage is calculated by Equation: \eqref{trans}.

\begin{equation}
\label{trans}
Vo = I*R_f
\end{equation}

*Add transistor in increase the Op Amp's current drive capability.
*Change capacitors in feedback loop.
*Add Sallen-key filter to output
*Explain difference in two buffering stages


The output will be buffered and then filtered and digitized dependent of the specific test of interest.



The current measurement circuit utilizes a transimpedance amplifier designed from the reference circuit in \cite{steve_thesis}. It utilizes 3 switched, feedback stages to measure 9 decades of current (See Table: \ref{imeas_table}). The circuit creates a virtual ground at the negative terminal of the DUT and then uses one of the feedback paths to create a voltage for a digitizer. Using a low-side current measurement circuit is beneficial in this circumstance because it allows low voltage circuitry to condition and measure a signal without needing to see the high DC voltage on the positive terminal of the DUT. The output voltage is calculated by Equation: \eqref{trans}.

\begin{equation}
\label{trans}
Vo = I*R_f
\end{equation}

*Add transistor in increase the Op Amp's current drive capability.
*Change capacitors in feedback loop.
*Add Sallen-key filter to output
*Explain difference in two buffering stages


The output will be buffered and then filtered and digitized dependent of the specific test of interest.



The current measurement circuit utilizes a transimpedance amplifier designed from the reference circuit in \cite{steve_thesis}. It utilizes 3 switched, feedback stages to measure 9 decades of current (See Table: \ref{imeas_table}). The circuit creates a virtual ground at the negative terminal of the DUT and then uses one of the feedback paths to create a voltage for a digitizer. Using a low-side current measurement circuit is beneficial in this circumstance because it allows low voltage circuitry to condition and measure a signal without needing to see the high DC voltage on the positive terminal of the DUT. The output voltage is calculated by Equation: \eqref{trans}.

\begin{equation}
\label{trans}
Vo = I*R_f
\end{equation}

*Add transistor in increase the Op Amp's current drive capability.
*Change capacitors in feedback loop.
*Add Sallen-key filter to output
*Explain difference in two buffering stages


The output will be buffered and then filtered and digitized dependent of the specific test of interest.



The current measurement circuit utilizes a transimpedance amplifier designed from the reference circuit in \cite{steve_thesis}. It utilizes three switched, feedback stages to measure nine decades of current (See Table: \ref{table:imeas_table}). The circuit operates by creating a virtual ground at the negative terminal of the \gls{dut}, and then directing the current through one of its feedback paths. The resultant voltage in Equation: \eqref{equ:trans} is then filtered and sent to the next stage for further signal conditioning.

\begin{equation}
\label{equ:trans}
Vo = I*R_f
\end{equation}

The AD817 op-amp has the precision and bandwidth needed to operate in a transimpedance amplifier configuration, but does not have the required drive strength for the  high current range. This limitation is overcome by inserting a BJT based current booster in the feedback path. This preserves the benefits of the op-amp, while providing the ability to drive a high-current signal.

