\subsection{Current Measurement}

The centerpiece of this part of the circuit is the transimpedance amplifier. It works by creating a virtual ground on the negative potential side of the DUT and then returning the current through a resistor to ground. This functionality is specifically useful in this application because it effectively uses the DUT to buffer the measurement circuitry from the high voltage. The transimpedance amplifier has 3 switched feedback stages that allow it to measure 9 decades of current. The output will be buffered and then filtered and digitized dependent of the specific test of interest.

The output voltage is calculated by Equation: \eqref{trans}.

\begin{equation}
\label{trans}
Vo = I*R_f
\end{equation}

The three measurement sections allow for a high precision of measurement over a wide range of current. They are listed in Table: \ref{imeas_table}.

\begin{table}[ht!]
\centering
\begin{tabular}{|l|l|l|}
\hline
Value & Range & Current   \\ \hline
5     & Hi    & 1A  - 1mA \\ \hline
5k    & Med   & 1mA - 1uA \\ \hline
5M    & Low   & 1uA - 1nA \\ \hline
\end{tabular}
\caption{Current Measurement Ranges}
\label{table:imeas_table}
\end{table}


