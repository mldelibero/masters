\subsection{Optocoupler}
\label{sec:opto}

The optocoupler \hyperlink{sch:opto}{circuitry} is based upon IXYS's application note AN-107 \cite{locAppNote}, and its purpose is to inject an AC signal ontop of a high DC bias. The optocoupler is configured in what is known as photovoltaic mode \cite{locAppNote}[Sec: 3.3], with both photoreceivers acting as coupled current sources.

In this mode, the front end op-amp's feedback path causes the first photoreceiver to draw current through the first resistor such that both of its terminals stay at ground potential. The phototransmitter's resistor sets a full scale current of 20mA, which is well within the absolute maximum ratings of the device. The second phototransmitter pulls current through the back end op-amp's feedback resistor with a voltage gain of 2.5.

The output of the optocoupler is referenced to the DC bias voltage from the SRS-PS350 instead of circuit ground. The limiting isolation specification comes from the $\pm 15V$ supply at 1.5kV, far below the 500V limit imposed by the system. The input sine wave is fed into the magnitude and phase measurement circuitry seen on Schematic Page: 7. The optically coupled sine wave is fed into the \gls{dut} as seen on Schematic Page 5.

