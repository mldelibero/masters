\section {Measurement Circuitry}
\label{sec:measCirc}
This section describes the schematic design, see Appendix: \ref{app:schematic}, used to meet the goals set out in the \nameref{sec:abstract}. At the time of this writing, this circuit has not been constructed, so what follows will be a theoretical explanation of the circuit's operation. The design builds on the capacitor anodization work done under the ARPA-E contract DE--AR000016 \cite{steve_thesis}.

The purpose of this circuit is to measure the response of a capacitor to various test inputs in order to generate a model (Section: \ref{sec:regression}). This method will allow an analysis of the effect of the DC bias on the output model.

This circuity provides for two main testing modes. In the first mode, a large (up to 500VDC) bias voltage is coupled onto an AC signal via an optoculpler. This voltage is then applied to the \gls{dut} and the resulting current flow is measured through a transimpedance amplifier. The signal is then filtered and its magnitude and phase are measured and digitized. The user has control of the DC bias, AC amplitude, and frequency. These control parameters can be swept to obtain a characterization of the \gls{dut}.

The second test mode provides a means to measure the discharge characteristics of the device. A DC bias is used to charge up the \gls{dut} and then disconnected. The circuity can either connect a discharge capacitor to the \gls{dut} or leave it open in order to measure the discharge current.


\nocite{my_ieeePaper}

\subsection {Power}
The power supply section, shown on Schematic Page: 2, shows the complete power generation scheme used for the circuit board. 24VDC is fused and then fed through an EMC compliance filter \cite{5VswitchDatasheet}. The initial fuse value is set to roughly twice the current rating of the switching power supplies. Once a practical current draw value is obtained, this value can be decreased. In order to ensure that the 3 switching regulators meet their rated regulation specifications, output shunt resistors are used to set a minimum load of $10\%$. All of the power supplies were chosen such that their power ratings were at least twice the maximum, expected load. The power supply generating $\pm 15V_{ISO}$ is slightly different than the rest, as its output ground is attached to the DC bias voltage instead of signal ground. This purpose of this seen in the optocoupler circuitry described in Section: \ref{sec:opto}.  


\subsection {DC Bias}
\label{sec:dcBias}

The DC bias signal is externally generated by a Stanford Research Supply SRS-PS350 power supply. Some of the models of this supply provide a serial input which can be used to set their output voltage. This scenario is preferred and the communications for it are handeled through the RS232 transceiver found on Schematic Page: 3.

\begin{figure}
\includegraphics[keepaspectratio=true,width=6in]{./figures/board/srsCtrl.jpg}
\centering
\caption{SRS Control Circuit}
\label{srsCtrl}
\end{figure}


Other models of this supply only provide an analog input option to control their output voltage. The specifications for this method can be seen in Table: \ref{table:srsCtrl_table}. The control signal is generated by using the microcontroller to set the DAC output over a SPI bus. This signal is then buffered and clamped to under 1.5V. Since the SRS-PS350 has a gain of 500, this clamps its output to 750VDC. The circuit is only designed to operate at up to 500VDC, but this value is still within the specified ratings of the components in the signal path.

The DAC has a 2.5V internal reference with a 14 bit resolution, giving the control signal an ideal resolution of $150\mu V$ and the SRS-PS350 an ideal output resolution of $76mV$. The update rate of this device is not important to this application because the DAC will be set infrequently and only when the system is not actively collecting data.

If different supply characteristics are needed, the SRS-PS340 can be swapped out with a different supply, but the control logic may not be supported.


\subsection{Optocoupler}
\label{sec:opto}

This section describes the circuitry shown on Schematic Page: 7. It is based upon IXYS's application note AN-107 \cite{locAppNote}, and its purpose is to inject an AC signal ontop of a high DC bias. The optocoupler is configured in what is known as photovoltaic mode \cite{locAppNote}[Sec: 3.3], with both photoreceivers acting as coupled current sources.

In this mode, the front end op-amp's feedback path causes the first photoreceiver to draw current through the first resistor such that both of its terminals stay and ground potential. The phototransmitter's resistor sets a full scale current of 20mA, which is well within the absolute maximum ratings of the device. The second phototransmitter pulls current through the the backend op-amp's feedback resistor with a voltage gain of 2.5.

The output of the optocoupler is referenced to the DC bias voltage from the SRS-PS350 istead of to circuit ground. The limiting isolation specification comes from the $\pm 15V$ supply at 1.5kV, far below the 500V limit imposed by the system. The input sine wave is fed into the magnitude and phase measurement circuitry seen on Schematic Page: 8. The optically coupled sine wave is fed into the DUT as seen on Schematic Page 4.


\subsection{Charging Circuitry}
\label{sec:charging}

This section describes the DUT charging circuitry seen in Schematic Page: 4. This part of the circuitry is meant to only be used to prepare the DUT for a test. The current is purposely limited in order to minimize the load on the SRS-PS350. The charging operation functions by enabling the high-current measurement and charging relays, and then slowly ramping the DC bias voltage until the DUT is fully charged.


\subsection{Discharging Circuitry}

The discharging \hyperlink{sch:discharging}{circuitry} provides a means to determine the bulk capacitance value of the \gls{dut} from the RC time constant. Once the \gls{dut} is charged via the operation described in Section: \ref{sec:charging}, the charging relay will open and the high-current discharge relay will close. The transimpedance amplifier described in Section: \ref{sec:iMeas} will measure the current through the \gls{dut} as it decays.

There are 3 switched discharge stages that allow for the decay rate to be regulated. Each stage allows for a range of voltages and capacitances needed to stay within safety and operational constraints. As seen in Figure: \ref{fig:opArea}, there are two constraints on both the voltage and capacitance. The maximum voltage is constrained by either the power rating on the lowest valued resistor or the overall safety limit of $500$VDC. The minimum capacitance value is constrained by setting a $100$ \gls{adc} samples per time constant with a sampling rate of $240$KSPS.

\begin{figure}
\includegraphics[keepaspectratio=true,width=3in]{./figures/circEx/operatingArea.jpg}
\centering
%    \cite{capSite_df_vs_temp}
\caption{Operating Area}
\label{opArea}
\end{figure}



\begin{table}[ht!]
\centering
\begin{tabular}{|l|l|l|}
\hline
Value & Range & Current   \\ \hline
5     & Hi    & 1A  - 1mA \\ \hline
5k    & Med   & 1mA - 1uA \\ \hline
5M    & Low   & 1uA - 1nA \\ \hline
\end{tabular}
\caption{Current Measurement Ranges}
\label{table:imeas_table}
\end{table}

\subsection{Filtering}
\label{sec:filtering}
\nocite{kirk_skey}

The filtering \hyperlink{sch:filtering}{circuitry} implements two filtering paths which each consist of a 6-pole, low-pass, Sallen-Key filter to condition the signal.

\begin{figure}[ht!]
\includegraphics[keepaspectratio=true,width=3in]{./figures/circEx/skey.jpg}
\centering
\caption{Low-Pass Sallen-Key Filter~\cite{skey}}
\label{fig:skey}
\end{figure}


\loadeq{skfc}
\loadeq{skq}

A generic, single-stage, low-pass, Sallen-Key filter is shown in Figure: \ref{fig:skey}. The filter is basically a 2-pole, passive, RC filter with the ground of $C_1$ tied to $V_{out}$ through the op-amp's active feedback path. This implementation is a simplified version of the Sallen-Key filter with the gain (listed as K in most explanations) equal to 1. The transition frequency and quality factor can be set according to Equations: \eqref{equ:sk_fc} and \eqref{equ:sk_q} respectively by choosing values for the passives.

\begin{figure}[ht!]
\includegraphics[keepaspectratio=true,width=3in]{./figures/circEx/skey.jpg}
\centering
\caption{Low-Pass Sallen-Key Filter~\cite{skey}}
\label{fig:skey}
\end{figure}


The filters for this circuit were designed with TI's WEBENCH designer \cite{webench}. Table: \ref{table:skey} shows the specifications for each filter. 
The AC current measurement branch has a cutoff frequency of 100kHz. With the overall measurement frequency range capped at 10kHz, this provides sufficient anti-aliasing for the measurement. The discharge measurement filter has a cutoff of 25kHz. With a $1k\Omega$ resistor and a 1uF \gls{dut} having a time constant of $10\mu s$, this provides a sufficient margin for 100 points per time constant requirement while still filtering out higher frequency components.


\subsection{Magnitude}
\label{sec:mag}

The magnitude \hyperlink{sch:imph}{circuitry} measures both the magnitude of the input sine wave and the magnitude of the resultant current waveform through the \gls{dut}. The design for this circuit is based on Analog Device's application note for the AD8277 \cite{absCircuit}.

The function of this circuit is to perform a full-wave rectification operation on the input signal and then pass the output to be filtered and then sampled. The first difference amplifier is configured as a voltage follower. Since it is only powered from a single rail, it passes the positive half of the waveform and shunts the negative half to ground. The second difference amplifier operates in different modes, based upon the output of the first differential amplifier. During the negative going portions of the waveform, its positive terminal is held at ground, and it operates as a unity gain, inverting amplifier. This rectifies the input signal to the output. During the positive going portions of the waveform, positive input terminal is held at the value of the input terminal. This forces the second difference amplifier to act as a voltage follower. The resultant waveform is a full-wave rectified version of the input signal. It is then bypassed with an output capacitor and fed into the \gls{adc} to calculate the value of the rms voltage.


\subsection{Phase}

The phase \hyperlink{sch:imph}{circuitry} measures both the phase of the input sine wave and the phase of the resultant current waveform through the \gls{dut}. It uses two, redundant phase measurement techniques.

The first technique utilizes the AD8302, a \gls{rf} phase detector, in a low frequency configuration as per AN-691 \cite{AD8302_AppNote}. The capacitors on the IC inputs tune the front end high pass filter cutoff to roughly 20Hz. The phase measurement is accomplished internally by a magnitude independent phase multiplier configuration. The output conisists of a 10mV per degree signal that is lowpassed at 2Hz to prevent aliasing. The filtered signal is then fed to the ADC for digitization.

The second technique is added for a low frequency comparison against the \gls{rf} phase detector. The comparator acts as a sine to square wave converter. The default configuration relies on the comparator's internal hysteresis, but provisions for wider thresholds are available if needed. The outputs of the two converters are fed into timer inputs on the microcontroller. The phase is then calculated by the \gls{mcu} by measuring the period of the signals and the time difference between their rising edges.


\subsection{ADC}

The \gls{adc} \hyperlink{sch:adc}{circuitry} makes use of the AD7656, a 6 channel, 250ksps, 16bit, SAR \gls{adc}. It sends the converted values to the \gls{mcu} for further processing and analysis via 3 parallel SPI outputs.


\subsection{Communications}

The communications \hyperlink{sch:com}{circuitry} is used to communicate to off-board processors.

\subsubsection{USB}
The \gls{usb} section is used to communicate with a PC for data logging and post processing. This circuitry centers around a FTDI FT232H serial to \gls{usb} chip. It allows seamless \gls{usb} communication to a PC's COM port with the \gls{mcu} only using a single UART port. This significantly lowers the complexity of the communication bus, as the \gls{mcu} is not responsible for running the \gls{usb} stack.

\subsubsection{RS-232}
The board also has two, bidirectional RS-232 ports. They are able to be used for general purposes, but will most often be used for communicating with a sine-wave generator and the DC Bias supply.



