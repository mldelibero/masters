\subsection{Magnitude}

The magnitude \hyperlink{sch:imph}{circuitry} measures both the magnitude of the input sine wave and the magnitude of the resultant current waveform through the \gls{dut}. The design for this circuit is based on Analog Device's application note for the AD8277 \cite{absCircuit}.

The function of this circuit is to perform a full-wave rectification operation on the input signal and then pass the output to be filtered and then sampled. The first difference amplifier is configured as a voltage follower. Since it is only powered from a single rail, it passes the positive half of the waveform and shunts the negative half to ground. The second difference amplifier operates in different modes, based upon the output of the first differential amplifier. During the negative going portions of the waveform, its positive terminal is held at ground, and it operates as a unity gain, inverting amplifier. This rectifies the input signal to the output. During the positive going portions of the waveform, positive input terminal is held at the value of the input terminal. This forces the second difference amplifier to act as a voltage follower. The resultant waveform is a full-wave rectified version of the input signal. It is then bypassed with an output capacitor and fed into the \gls{adc} to calculate the value of the rms voltage.

