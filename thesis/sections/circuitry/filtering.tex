\subsection{Filtering}
\label{sec:filtering}
\nocite{kirk_skey}

This section describes the current filtering circuitry shown in Schematic Page: 6. Both filtering paths implement a 6-pole, low-pass, Sallen-Key filter to condition the signal.

\begin{table}[ht!]
\centering
\begin{tabular}{|l|l|l|l|l|}
\hline
{\bf Filter}         & {\bf Filter Order} & {\bf Gain} & {\bf f\_c} & {\bf Stopband Attenuation} \\ \hline
{\bf Discharging}    & 6                  & 1          & 25kHz      & -65dB                      \\ \hline
{\bf AC Measurement} & 6                  & 1          & 100kHz     & -100dB                     \\ \hline
\end{tabular}
\caption{Sallen-Key Filter Specifications}
\label{table:skey}
\end{table}



\begin{equation}
    \label{equ:sk_fc}
    f_c = \frac{1}{2\pi R_2C_2\sqrt{\frac{R_1C_1}{R_2C_2}}}
    ~\cite{hh}[ch~6.3.2.D~equ~6.5~pg~410]
\end{equation}

\begin{equation}
    \label{equ:sk_q}
    Q = \frac{\sqrt{\frac{R_1C_1}{R_2C_2}}}{1 + \frac{R_1}{C_1}}
    ~\cite{hh}[ch~6.3.2.D~equ~6.6~pg~410]
\end{equation}

A genaric, single-stage, low-pass, Sallen-Key filter is shown in Figure: \ref{fig:skey}. The filter is basically a 2-pole, passive, RC filter with the ground of $C_1$ tied to $V_{out}$ through the op-amp's active feedback path. This implementation is a simplified version of the Sallen-Key filter with the gain (listed as K in most explanations) equal to 1. The transition frequency and quality factor can be set according to Equations: \eqref{equ:sk_fc} and \eqref{equ:sk_q} respectively by choosing values for the passives.

\begin{table}[ht!]
\centering
\begin{tabular}{|l|l|l|l|l|}
\hline
{\bf Filter}         & {\bf Filter Order} & {\bf Gain} & {\bf f\_c} & {\bf Stopband Attenuation} \\ \hline
{\bf Discharging}    & 6                  & 1          & 25kHz      & -65dB                      \\ \hline
{\bf AC Measurement} & 6                  & 1          & 100kHz     & -100dB                     \\ \hline
\end{tabular}
\caption{Sallen-Key Filter Specifications}
\label{table:skey}
\end{table}



The filters for this circuit were designed with TI's WEBENCH designer \cite{webench}. Table: \ref{table:skey} shows the specifications for each filter. 
The AC current measurement branch has a cutoff frequency of 100kHz. With the overall measurement frequency range capped at 10kHz, this provides sufficient anti-aliasing for the measurement. The discharge measurement filter has a cutoff of 25kHz. With a $1k\omega$ resistor and a 1uF \gls{dut} having a timeconstant of $10\mu s$, this provides a sufficient margin for 100 points per time constant requirement while still filtering out higher frequency components.

