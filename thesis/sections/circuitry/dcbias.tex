\subsection {DC Bias}

A DC bias is needed to conduct AC measurements as well as to allow for measuring discharge curves of the DUT. In order to determine the effect of the DC bias on the DUT, it is necessary to control that quantity. This section describes the circuitry (Schematic Page: 3) used to provide a means to programatically control the DC bias from 0 to 500V. The supply used is a Stanford Research Supply (SRS) PS350. It accepts a DC control signal, which is supplied by the circuit through a DAC. If different supply characteristics are needed, it can be swapped out with any supply. A generic RS232 bi-directional port (Schematic Page: 10) is provide to control the supply in this situation.

\begin{table}[ht!]
\centering
\begin{tabular}{| l | l | }
\hline
Input Scale      & 0 to +10V for 0 to 5kV                          \\ \hline
Input Impedance  & 1 M$\Omega$                                     \\ \hline
Accuracy         & $\pm 0.2\%$ of full scale                       \\ \hline
Update Rate      & 15 Hz                                           \\ \hline
Output Slew Rate & \textless 0.3s for 0 to full scale (full load)  \\ \hline
\end{tabular}
\caption{SRS-PS350 Analog Control Characteristics \cite{srsManual}\cite{srsCatalog}}
\label{table:srsCtrl_table}
\end{table}


The SRS Supply's specifications can be seen in Table: \ref{srsCtrl_table}. The output of the PS350 can either be set manually or with a 0-10V control voltage. This circuitry will take advantage of this analog input at a gain of 500. 

The analog control uses an AD7391, 14-bit DAC with an internally generated reference. It is controlled via a SPI bus connected to a microcontroller. The update rate of this device is not important to this application because it will be set infrequently and only when the system is not actively collecting data. The output signal of the DAC is fed into an op amp which acts as a voltage limiter for safety. This clamps the control signal at 1.5V and the PS350's output at 750V.These numbers are theoretical and will degrade from ideal due to the SRS's response to the control signal, the noise levels in the system, and various other factors, such as temperature.

The DAC has a 2.5V internal reference. With 14 bits, that gives the control signal an ideal resolution of $150\mu V$ and the PS350 an output resolution of $76mV$.

