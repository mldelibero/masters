\section {Conclusion}
\label{sec:conclusion}

\ifisPPT
\begin{frame}
    \frametitle{Conclusion}

Circuit Capabilities:
\begin{itemize}
    \item Discharge Curve
    \item Impedance
\end{itemize}

Measurement Range:
\begin{itemize}
    \item DC bias range 0$\rightarrow$500V.
    \item Frequency Range of 100Hz$\rightarrow$40kHz.
\end{itemize}

Regression Accuracy:
\begin{itemize}
    \item Accurrate outside of resonance.
    \item $2 \Omega$ and $2^{\circ}$.
    \item $<5\%$.
\end{itemize}
\end{frame}
\else
In conclusion, the circuitry and regression analysis developed in this thesis shows a promising ability to accurately model capacitors. The method shown provides the ability to model a capacitor over a 0 $\rightarrow$ 500VDC bias and a 100Hz $\rightarrow$ 40kHz range. The regression analysis provides a means to compare new capacitors with unknown characteristics. The accuracy of the model is mostly determined by the number of parameters and typically has the greatest relative error near resonance. Outside of resonance and low frequencies, the accuracy of the fit stays within $2 \Omega$ and $2^{\circ}$. The relative error, outside of a single peak in the phase plot, stays well below $5\%$. This method provides a good estimate of titanium capacitor parameters, which are best suited for power applications  at low frequencies.
\fi

