\section {Background}

The following is a list of the questions that I will be answering in my background section.

\subsection{List}
\begin {enumerate}
    \item What are the main failure modes?
    \item What are the current specifications of the parts in use now?
    \item What research is being done to develop better capacitors for this use case?
    \item How do they currently evaluate the capacitors?
    \item How would they benefit from my research?
    \item What is the state of the art in capacitance measurement?
    \begin {enumerate}
        \item Impedance analyzers
        \item Capacitance bridges
        \item Hi pot testers
        \item What are the good and bad points of each of these technologies?
        \item Why do they not solve the problem that I stated? 
        \item Why does this technology not currently exist?
    \end {enumerate}
    \item What work has been done similar to this in the past?
    \item What are the important characteristics of capacitors?
    \item Is there any evidence that a capacitor's properties will change over DC bias?
\end {enumerate}

\subsection {History of Capacitors}
\label{sec:history}

This section will chronolog the history of capacitors. It will link various introductions in the technology to advances in industry, and it will map the driving forces behind capacitor development.

Capacitors have their origin in the invention of the Leyden jar by Peter van Musschenbroek of Leiden University in 1745 \cite{empLight}. Before this point, scientists were able to generate static electricity through electrostatic machines, but had a very limited ability to store this electrical energy \cite{ieee_hist}.
The most common design for the Leyden jar was to use a glass jar with metal foil lining the inside and out. The inner foil was typically charged with an electrostatic generator, while the outer foil was connected to ground. The charge would stay on the metal foil until a short or small resistance was connected between them. Charge could be stored this way, allowing scientists and showmen, to use greater amounts of current than they could generate at any one moment.
Since the Leyden jar, many different types of capacitors have risen and fallen in prominence in the market. This section covers the historical introduction of some of the major types.

In 1876 Fitzgerald introduced wax impregnated paper dielectric capacitors with foil electrodes \cite[ch.~11]{dumInv}\cite{learn_caps}. They were typically used for power supply filtering in radios. By the early 1920s, they existed as tubes encapsulated in plain, Bakelite cardboard, with bitman sealing the ends and waxed paper as a dielectric \cite[ch~3]{dumInv}.
Paper capacitors were upgraded to impregnated paper capacitors, which used paper that was soaked in mineral oil. They were interleaved with metal foil and then rolled to make the capacitor \cite[ch.~8.2.1.1]{poorIntro}. During WWII, paper capacitors were upgraded with metal-cased tubes with a rubber end \cite[ch.~8.1]{poorIntro}. These metalized paper capacitors were constructed similarly with the improvement that one side of the paper dielectric was sprayed with metal \cite{hist_cerFilt}.

Karol (Charles) Pollak discovered the principle of the electrolytic capacitor in 1886 while he was researching the anodization of metals, and received a patent for the borax-solution aluminum electrolytic capacitor in 1897.
In 1936 Cornell-Dubilier opened a factory to produce aluminum electrolytic capacitors.
After the start of WWII, increased funding and effort was applied to the cause of electrolytic capacitors and techniques such as ``etching and pre-anodizing'' greatly increased their reliability \cite{deis_hist}\cite{wiki_elec}.

In 1957 GE extended the science of electrolytic capacitors into what is now known as electrochemical capcaitors. This technology, also known as electric double layer capacitors, was not commercialized until NEC licensed Standard Oil's patent in 1978 \cite{electrochem_intro}. This ``supercapacitor'' formed the basis of today's electrochemical capacitors. They are characterized by a large capacitance in the range of kilo farads, but with a working voltage of only 2.7V \cite{electrochem_intro}. Some of today's typical applications for EC capacitors are in battery replacement, electric vehicles, and regenerative breaking.

\nocite{hh}
\nocite{capGuide_mica}
M. Bauer of Germany invented the mica capacitor in 1874. The original mica capacitor was a ``clamped'' style capacitor, which was used through the 1920s \cite{wiki_mica} and then replaced by silver mica capacitors \cite{learn_caps}.
Mica's inherent inertness and reliability allowed for extreme reliability and efficiency in a packaged capacitor \cite{tedds_mica}. Mica capacitors were heavily used in the radio industry due to their superb stability at RF frequencies and their physical robustness \cite{radio_mica}.
Capacitors created with mica allowed for a comparatively smaller product \cite[f.~37-41]{dumInv} with the ability to survive shock from weapons better than its glass counterpart. Consequently, mica capacitors began to be produced in large quantities during WWI.
In light of mica supply chain problems and the emergence of ceramic capacitors during WWII, mica capacitors fell from prominence to a niche market \cite[Ch 3, Sec II]{cerMaterials}.

The first glass ceramic dielectric capacitor was the Leyden jar. While this early capacitor was used mainly for scientific experiments, commercial glass capacitors came later.
Glass tubular capacitors, known as Moscicki tubes, appeared in 1904 and were used in Marconi's experiments in wireless transmission. They continued to be used in wireless communication until about WWI \cite[p.~102]{dumInv}.

Scientists in Germany created the first steatite ceramic capacitors in 1920 \cite[Ch 3 Sec II]{cerMaterials}\cite{cerDie}. Also known as talc, this ceramic capacitor variant was able to closely match the temperature coefficient of mica \cite{steatite_hf}. Rutile was later introduced into ceramic capacitor technology. On its own, it produced a dielectric constant of 10 times that of steatite, but the two were typically blended to get a better temperature coefficient.

A ceramic composition with barium titanate ($BaTiO_3$) was first discovered in 1941. Barium titanate was quickly found to be able to exhibit a dielectric constant over 1000; an order of magnitude greater than the best capacitor at that time (rutile - $Ti0_2$). It was not until 1947, that barium titanate appeared in its first commercial devices, phonograph pickups \cite{piezCer}\cite{hist_cerFilt}\cite[Ch 3 Sec III]{cerMaterials}. Barium titanate is still used today in certain types of \glspl{mlcc}.

Thanks to the semiconductor industry, \glspl{mlcc} grew to dominate the capacitor market and by 1979, AVX alone reached $\$95$ million in \gls{mlcc} sales \cite{avx_hist}.
\glspl{mlcc} are divided up into three classes. Class 1 type \glspl{mlcc} are known for their extremely good temperature characteristics. COG/NPO types can have 0-30ppm/$^o$C. These capacitors are typically made by combining $TiO_2$ with additives in order to adjust its temperature characteristics \cite{intro_cerCaps}. Additionally, class 1 \glspl{mlcc} have comparatively poor volumetric efficiency, and will tend to come in larger packages than class 2 \glspl{mlcc} of the same capacitance value. Class 2 types typically have worse temperature coefficients than class 1 types, but they have a much higher volumetric efficiency. They are constructed with a ferroelectric base material, typically barium titanate \cite{intro_cerCaps}. Class 2 \glspl{mlcc} are the most common type of ceramics used. Class 3 types have very high capacitance, but a working voltage of several volts \cite{hist_cerFilt}\cite[Ch 3 Sec VI]{cerMaterials}\cite{atCer_tempco}. They were originally developed as a potential replacement for liquid electrolytics, but they have largely been replaceed by Class 2 \glspl{mlcc} \cite{wiki_cer}. The advances in Class 2 \glspl{mlcc} in maximum capacitance shrunk the gap between these two classes until the difference was no longer viable \cite{wiki_cer}.

Bell labs invented the first solid tantalum capacitor in 1956. They created it in conjuncture with, and for, transistors \cite[f.~56-64]{dumInv}. Tantalum capacitors typically have better characteristics than aluminum electrolytics, but have a lower maximum capacitance and working voltage \cite{learn_caps}.
Sprague patented the first commercially viable solid tantalum capacitor in 1960. It offered an increased capacitance per unit volume and greater reliability \cite{charTant}. In the 1970s, Sprague released the first surface mount tantalum capacitor \cite{spragueHist}.
One of the historical problems with tantalum capacitors has been a limited and volatile supply of tantalum in the world market.
As a result of a price spike around 1980, the industry reacted by creating finer grain tantalum powders. This allowed a unit to be made with less overall tantalum, reducing price and package size \cite[ch~3.1]{tantMis}. Over time, this volatility resulted in better tantalum capacitors, but decreased their popularity and led to a desire for a suitable replacement.

One of the possible alternatives to tantalum capacitors is a titanium based capacitor. Some of the instrumental work in titanium alloys was done in 1939 by Fast \cite{fast1939transition} and more recently in 1995 by Kobayashi \cite{kobayashi1995mechanical}\cite{corwin2013synthesis}. Early work on titanium based capacitors resulted prohibitively high leakage currents \cite{ki2005titanium}, but research done by Welsch \cite{welsch22nd}\cite{welsch2001directionally} with titanate ($TiO_2$) shows a promising a solution to this issue. This work also indicates that a Ti capacitor can be made with a much higher energy and power density when compared with tantalum \cite{steve_thesis}.


\subsection{Ti Capacitors}

Even though the work in this thesis can characterize wide range of components, its primary focus is on titanium capacitors. As stated in the previous section, titanium capacitors have many advantages over their primary competitor, tantalum capacitors. They have fewer sourcing issues, as titanium is both cheaper and can be mined in conflict free zones. Also, the higher power and energy density, along with a high working voltage, make titanium capacitors an attractive option. These characteristics allow for a more efficient product, that is both cheaper and lighter than tantalum capacitors.  The main reason for why titanium capacitors do not have the market share of tantalum capacitors is that they have been plagued with high leakage currents. The initial work by Ehret\cite{steve_thesis} on the ARPA-E DE--AR000016 contract focused on measuring this quantity. The circuitry designed in this thesis (Section: \ref{sec:measCirc}) expands upon that design to also measure additional characteristics.

One of the main potential applications for titanium capacitors is the DC-link capacitor for electric vehicles \cite{dclink}. Its purpose is to provide an energy source and storage mechanism that has a higher power density that the high energy density batteries in the vehicle \cite{thounthong2006control}. They allow for a high surge of power when the vehicle is accelerating, and also allow the vehicle to recapture some of the expended energy through regenerative breaking.

Another potential use of titanium capacitors is in modern defibrillators. These devices require capacitors that can deliver a quick burst of energy while operating at no less than several hundred volts \cite{shephard1961design}\cite{idiot_defibrillator}. One of the main requirements of the capacitor in defibrillators is extremely high reliability. The titanium capacitors proposed by Welsch \cite{welsch22nd} provide an attractive option here because of their self-healing characteristic. Small defects in the dielectric are automatically healed, allowing the capacitor to continue to function while minimize failure events. 

