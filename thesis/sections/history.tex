\section {History of Capacitors}
This section will chronolog the history of capacitors. It will link various introductions in the technology to advances in industry, and it will map the driving forces behind capacitor development.

\noindent Types of capacitors:
\begin{enumerate}
    \item Leyden jars
    \item ceramic 
    \item aluminum electrolytic 
    \item tantalum 
    \item polymer
    \item metalized film
    \item foil film
    \item titanium 
\end {enumerate}

\noindent Driving forces for development:
\begin {enumerate}
    \item scientific study
    \item military
    \item radio industry
    \item consumer electronics
\end {enumerate}

\noindent Significant players:
\begin{enumerate}
    \item ...
\end {enumerate}

\subsection{Leyden Jar}


\noindent Capacitors have their origin in the invention of the Leyden jar by Peter van Musschenbroek of Leiden University in 1745. \cite{empLight} It allowed for the storage of electrical energy for the first time in known history. 

\noindent The Leyden jar was typically made from a glass jar with metal sheets spread on the inside and outside. Electricity could be stored by charging the jar with an electrostatic device and the removing the jar. This breakthrough in the study of electricity was exremely important to scientists, as it allowed electricity to be stored and then used later.
\cite{ieee_hist} 

\noindent The most common design for the Leyden jar was to use a glass jar with metal foil lining the inside and out. Then inner surface was typically charged via an electrostatic generator, while the outer surface was connected to ground. The charge would stay on the metal foil until a short or small resistance was connected between them. Charge could be stored this way, allowing scientists, and showmen, to use greater amount of charge than they could generate at any one moment.

\noindent Since the Leyden jar, many different types of capacitors have risen and fallen in prominence in the market. This section will cover the historical introduction of some of the major types.

\noindent Paper capacitors use waxed paper as a dielectric. They are primarily used in high voltage applications. But they are not preferred for much due to their high leakage and tolerances.\cite{learn_caps} In 1876 Fitzgerald introduced wax impregnated paper dielectric capacitors with foil electrodes.\cite[ch.~11]{dumInv} \cite{learn_caps} They were typically used for power supply filtering in radios. In the early 1920s, they existed as tubes encapsulated in plain, bakalized cardboard, with bitman sealing the ends.\cite[ch~3]{dumInv}

\noindent Impregnated paper capacitors used paper that was soaked in mineral oil. It was interleaved with metal foil and then rolled to make the capacitor.\cite[ch.~8.2.1.1]{poorIntro} \noindent During WW2, paper capacitors were upgraded with metal-cased tubes with a rubber end.\cite[ch.~8.1]{poorIntro} Metalized paper capacitors were created as a replacement for impregnated paper capacitors. They were constructed similarly with the improvement that one side of the paper dielectric was sprayed with metal.\cite{hist_cerFilt}


\nocite{hh_cap_table}
\nocite{capGuide_mica}
\noindent M. Bauer of Germany invented the mica capacitor in 1874. The original mica capacitor was a "clamped" style capacitor, which was used through the 1920s.\cite{wiki_mica} Clamped style mica capacitors were eventually replaced by silver mica capacitors, which greatly increased mica capacitors' characteristics.\cite{learn_caps}.

\noindent Mica's inherent inertness and reliability allowed for extreme reliability and efficiency in a packaged capacitor.\cite{tedds_mica} The mica capacitor was heavily used in the radio industry due to its superb stability at RF frequencies and physical robustness.\cite{radio_mica}

\noindent During WWI, mica capacitors began to be produced in large quantities. This was mainly due to it being able to survive shock from weapons better than its glass counterpart. Also, it allowed the capacitors to be shrunk to achieve the same purpose. \cite[f.~37-41]{dumInv}

\noindent In light of mica supply chain problems and the emergence of ceramic capacitors during WW2, mica capacitors fell from prominence to a niche market.\cite[Ch 3, Sec II]{cerMaterials}

\noindent The first glass dielectric capacitor was, in fact, the Leyden jar. While this early capacitor was used mainly for scientific experiments, commercial glass capacitors came later. 

\noindent Glass tubular capacitors appeared in 1904 and were used in Marconi's experiments in wireless transmission. They were know as Moscicki tubes. They continued to be used in wireless communication until about WWI. \cite[p.~102]{dumInv}

\noindent Scientists in Germany created the first steatite ceramic capacitor in 1920. \cite[Ch 3 Sec II]{cerMaterials} \cite{cerDie} Also known as talc, this ceramic capacitor variant was able to closely match the temperature coefficient of mica.\cite{steatite_hf}     

\noindent Rutile ceramic capacitors were introduced with a dielectric constant of 10 times that of steatite. It was typically blended with steatite to get a better temperature coefficient. 

\noindent A ceramic composition wth barium titanate (BaTiO3) was first discovered in 1941. Barium titanate was quickly found to be able to exhibit a dielectric constant over 1000; an order of magnitude greater the best at this time (rutile - $Ti0_2$). It was not until 1947, that barium titanate appeared in its first commercial device, phonograph pickups.\cite{piezCer}\cite{hist_cerFilt}\cite[Ch 3 Sec III]{cerMaterials} Today barium titanate is still seen in multi layer ceramic capacitors.

\noindent Multi-layer ceramic capacitors are the current leading technology in commercially produced units. They are divided up into three classes. Class 1 type MLCCs are known for their extremely good temperature characteristics. COG/NPO types can have 0-30ppm/$^o$C. Class 2 types typically have worse temperature coefficients than type 1 types, but they have a much higher volumetric efficiency. Class 3 types have very high capacitance, but a working voltage of several volts.\cite{hist_cerFilt}\cite[Ch 3 Sec VI]{cerMaterials}\cite{atCer_tempco}

\noindent Bell labs invented the first solid tantalum capacitor in 1956. They created it in conjuncture, and for, transistors.\cite[f.~56-64]{dumInv} Tantalum capacitors typically have better characteristics than aluminum electrolytics, but have a lower maximum capacitance and working voltage.\cite{learn_caps}

\noindent Sprague patented the first commercially viable solid tantalum capacitor in 1960. It offered an increased capacitance per unit volume and greater reliability.\cite{charTant} In the 1970s, Sprague released the first surface mount tantalum capacitor.\cite{spragueHist}

\noindent One of the historical problems with tantalum capacitors has been a limited supply of tantalum in the world market. This has occasionally caused price spikes in the material, hurting the tantalum capacitor market. As a result of the price spike around 1980, manufacturers created finer grain tantalum powders. This allowed a unit to be made with less overall tantalum, reducing price and package size.\cite[ch~3.1]{tantMis}
