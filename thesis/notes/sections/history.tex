\section {History of Capacitors}
\subsection{Leyden Jar}
Capacitors have their origins in the Leyden jar; developed by Ewald Jrgen von Kleist in 1745 and independently by Pieter van Musschenbroek in 1746. \cite{ieee_hist}

The Leyden jar was typically made from a glass jar with metal sheets spread on the inside and outside. Electricity could be stored by charging the jar with an electrostatic device and the removing the jar. This breakthrough in the study of electricity was exremely important to scientists, as it allowed electricity to be stored and then used later.
\cite{ieee_hist} 
---
The most common design for the Leyden jar was to use a glass jar with metal foil lining the inside and out. Then inner surface was typically charged via an electrostatic generator, while the outer surface was connected to ground. The charge would stay on the metal foil untill a short or small resistance was connected between them. Charge could be stored this way, allowing scientists, and showmen, to use greater amount of charge than they could generate at any one moment.

---
"At this stage, the pursuit of electrical knowledge was restricted by the limited amount of eletricity experimenters could atually generate."
\cite{empLight}

"First solution to storing electricity."
---
The discovery of the Leyden jar allowed scientists to store elecricity for the first time. \cite{wws_hist}
---

\subsection{Misc}
In 1872, the standard unit of measurement of a capacitor's capcitance was set to be the Farad. \cite{ieee_hist} 

In the 1950s, Bell labs began to popularize the transistor, which allowed circuits to become smaller and deal with lower voltages. This also lead to the development of smaller capacitors. \cite[f.~12]{dumInv}

Birth of some capacitors:

\begin{tabular} { l  r}
Name & Date \\
Leyden Jar   & 1745 \\
Mica         & 1874 \\
Paper        & 1876 \\
Ceramic      & 1900 \\
Electrolytic & 1922 \\
Glass Tub    & 1904 \\
Solid Electrolytic & 1956 \\
\end{tabular} \cite[f.~19-25]{dumInv}


Glass tubular capacitors (Moscicki tubes) were the only option available to Marconi in his early wireless experiments.\cite[f.~49-41]{dumInv}


---
In the 1950s, Bell labs began to popularize the transistor, which allowed circuits to become smaller and deal with lower voltages. This also lead to the development of smaller capacitors. \cite[f.~12]{dumInv}

\subsection{Ceramic Capacitors} \nocite{cerSurvey}
    \subsubsection{Basic Ceramic Dielectric Materials}  \cite[Ch 3 Sec IV]{cerMaterials}
        Materials:
        \begin{enumerate}
            \item Porcelein and Steatite \cite[Ch 3 Sec IV.A]{cerMaterials}
            \begin{enumerate}
                \item Early ceramic dielectrics. 
                \item alkaline oxide-free compositions?
                \item $\epsilon _r = [5,7]$
                \item positive tempco
            \end{enumerate}

        \end{enumerate}
        \cite[Ch 3 Sec IV]{cerMaterials}

        \subsubsection{Varieties of Ceramic Capacitors} \cite[Ch 3 Sec VII]{cerMaterials}
            \begin{enumerate}
                \item Film
                \begin{enumerate}
                    \item Thin-Film
                    \item Thick-Film
                \end{enumerate}
                \item Single-Layer discrete Capcitors
                \item Multilayer Capacitors
                \item Barrier Layer Capacitors
                \item Multilayer GBBL Capacitors
            \end{enumerate}


    \subsubsection{Misc}

        \begin{itemize}
            \item First one was the Leyden jar
            \item 1926 -- titanium dioxide (rutile) became available.
            \item 1941 -- barium titanate became available (eps = 1000)
            \item 1970-80s -- MLCC processes expanded ceramic capacitors \cite{deis_hist}
        \end{itemize}

        Ceramic capacitors have several advantages, some of which include a high capacitances per unit volume, and a low ESL. NPO ceramic capacitors have one of the most predictable temperature coefficients on the market. \cite{learn_caps}

        In light of mica supply chain problems and the emergence of ceramic capcitors during WW2, mica capacitors fell from prominence to a niche market after the second great war.\cite[Ch 3, Sec II]{cerMaterials}

        Ceramic capacitors have several advantages, some of which include a high capacitances per unit volume, and a low ESL. NPO ceramic capacitors have one of the most predictable temperature coefficients on the market. \cite{learn_caps}

        "
        pros:
        \begin{enumerate}
        \item high $\epsilon$
        \item low ESR
        \item good for SMD packaging
        \end{enumerate}

        Cons:
        \begin{enumerate}
        \item low breakdown voltage
        \end{enumerate}
        " \cite{capSite_intro}
        
    \subsubsection{Characteristics}
        Film Capacitors:
        \begin{itemize}
            \item Large swing in available energy storage
            \item high reactive power
            \item modest energy density
            \item high power density
            \item bipolar
            \item stable capacitance with frequency and voltage
            \item low ESR
            \item low ESL
            \item very low DF
            \item suitable for high voltage applications (<=100kV)
        \end{itemize}

        Typical Uses:
        \begin{itemize}
            \item moderate to large value capacitances
            \item dc and ac applications
            \item high power elctronics
            \item high frequency filtering
            \item continuous ac operation
            \item solid state switch snubbers
            \item SCR communication circuits
            \item PF correction
            \item motor start and run capacitors
        \end{itemize}
        \cite{capRev}

        Typical Uses:
        \begin{itemize}
            \item surface mount
            \item low voltage dc control circuits
            \item blocking
            \item buffering
            \item bypass
            \item coupling
            \item low frequency filtering
            \item tuning
            \item timing (<=GHzs)
        \end{itemize}
        \cite{capRev}

        CDL Characteristics:
        \begin{itemize}
            \item high energy density
            \item low power density
            \item small dc working voltage
        \end{itemize}

        Typical Uses:
        \begin{itemize}
            \item energy storage for hybrid vehicles
            \item reservoir caps for SMPS
            \item regenerative breaking
        \end{itemize}
        \cite{capRev}

    \subsubsection{Classes}
        \subsubsection{Class 1}
            \begin{itemize}
                \item ~Low capacitance
                \item low dielectric loss
                \item low tempco of capacitance -- 0 to +/- 75000 ppm$/^oC$
                \item most common tempco is 0 to +/- 30ppm/$^oC$
                \item low rate of aging
                \item temprange [$-55^oC,85^oC$]
                \item $\epsilon -r <= 100$
            \end{itemize} \cite[Ch 3 Sec VI]{cerMaterials}

            Class 1 ceramics, (TCC caps), have very good temperature coefficients, and are used when you need to have a precisely valued capacitance over wide operating temperatures.  \cite{atCer_tempco}

            Class 1:
            \begin{itemize}
                \item K = 5-100s
                \item DF $<<$.01
                \item Linear tempco 0-1000s ppm/$^oC$ \cite{hist_cerFilt}
            \end{itemize}

            Early Types:
            \begin{itemize}
                \item porcelain
                \item steatite
                \item mica
                \item misc silicates
            \end{itemize}
               

            Modern Types:
            \begin{itemize}
                \item rutile
                \item perovskite titanates \cite{hist_cerFilt}
            \end{itemize}

            \begin{table}
            \begin{tabular}{| l | l |  l |  l |  l |  l | }
            \hline
            \multicolumn{2}{|c|}{First Marking}                        &    \multicolumn{2}{c|}{Second Marking} &               \multicolumn{2}{c|}{Third Marking} \\ \hline
            Symbol                                          & Signifcant Digit (ppm/$^oC$) & Symbol       & Multiplier & Symbol        & Tolerance (ppm/$^oC$)     \\ \hline
            C                                               & 0.0                        & 0              & -1         & G             & +/-30                     \\ \hline
            B                                               & 0.3                        & 1              & -10        & H             & +/-60                     \\ \hline
            L                                               & 0.8                        & 2              & -100       & J             & +/-120                    \\ \hline
            A                                               & 0.9                        & 3              & -1000      & K             & +/-250                    \\ \hline
            M                                               & 1.0                        & 5              & +1         & L             & +/-500                    \\ \hline
            P                                               & 1.5                        & 6              & +10        & M             & +/-1000                   \\ \hline
            R                                               & 2.2                        & 7              & +100       & N             & +/-2500                   \\ \hline
            S                                               & 3.3                        & 8              & +1000      & ~             & ~                         \\ \hline
            T                                               & 4.7                        & ~              & ~          & ~             & ~                         \\ \hline
            V                                               & 5.6                        & ~              & ~          & ~             & ~                         \\ \hline
            U                                               & 7.5                        & ~              & ~          & ~             & ~                         \\ \hline
            \end{tabular}
            \caption{EIA Specification Codes for Class 1 Dielectrics} \cite{hist_cerFilt}
            \end{table}

        \subsubsection{Class 2}
            \begin{itemize}
                \item $\epsilon _r <= 15,000$ 
                \item The mostly ferroelectric barium titanate -- causes concerns with:
                \begin{itemize}
                    \item voltage coefficient of capacitance
                    \item voltage coefficient of series resistance
                    \item voltage coefficient of dissipation factor
                    \item rate of capacitance decrease upon aging.
                \end{itemize}
                \item Mostly Chosen for:
                \begin{itemize}
                    \item tempco of capacitance
                    \item temperature range
                \end{itemize}
            \end{itemize} \cite[Ch 3 Sec VI]{cerMaterials}

            Class 2 ceramics have slightly worse characteristics, but have a much higher volumetric efficiency.  \cite{atCer_tempco}

            Class 2:
            K = 1000 - 20,000
            ferroelectric ceramics
            DF = .01-.03
            mod-high temperature dependence \cite{hist_cerFilt}

        \subsubsection{Class 3}
            \begin{itemize}
            \item Very high capacitance
            \item Low working voltage
            \end{itemize} \cite[Ch 3 Sec VI]{cerMaterials}

            Class 3:
            barrier layers
            very high capacitance
            $<$ 25V working voltage \cite{hist_cerFilt}
    \subsubsection{Misc}
        Ceramic capacitors are completely inert; which means that they will not deteriorate under normal operating conditions.\cite[f.~45-46]{dumInv}

    \subsubsection{Glass (Porcelain)}
        The first glass dielectric capacitor was, in fact, the Leyden jar. While this early capacitor was used mainly for scientific experiments, commercial glass capacitors came later. 

        Glass tubular capacitors appeared in 1904 and were used in Marconi's experiments in wireless transmission. They were know as Moscicki tubes. They continued to be used in wireless communication until about WWI. \cite[p.~102]{dumInv}

    \subsubsection{Steatite}
        Germany 1920 -- steatite ceramics that used alkaline earth oxide fluxes instead of alkaline oxide (feldspare) \cite[Ch 3 Sec II]{cerMaterials}

        "Steatite, of which the principle constituent is magnesium silicate $(MgSiO_2)$ has a high resistivity and a temperature coefficient close to that of mica." \cite{cerDie}
        
        Magnesium silicate is known as talc. Talc existing in nature in large quantities is known as steatite. \cite{steatite_hf}     

    \subsubsection{Rutile}
            \begin{enumerate}
                \item First blended with Steatite
                \item x10 $\epsilon _r$
                \item lower +tempco or -tempco
                \item earliest know properties were single-crystal rutile in 1902.
                \item best ceramic mixture, steatite-$Ti0_2$:
                \begin{enumerate}
                    \item near zero tempco
                    \item $\epsilon _r = [15,20]$
                \end{enumerate}
                \item DF = [2,5]X low-loss steatite, but can be blended with steatite to get closer.\cite[Ch 3 Sec IV.B]{cerMaterials}
            \end{enumerate}

            1926 -- titanium dioxide (rutile) became available.\cite{deis_hist}

\subsection{Barium-Titanate Capacitors} \nocite{histFerr}
    \subsubsection{History}
        A ceramic composition wth barium titanate (BaTiO3) was first discovered in 1941. Barium titanate was quickly found to be able to exhibit a dielectric constant over 1000; an order of magnitude greater the best at this time (rutile - Ti02). It was not until 1947, that barium titanate appeared in its first commercial device, phonograph pickups.\cite{piezCer}

        "Barium titanate ceramics were discovered by E. Wainer and N. Saloman in the USA in 1942., by T. Ogawa in Japan in 1944, and by B. M. Vul in the soviet Union in 1944. All discoveries were and independently with no communication between the researchers because of World War II."\cite{hist_cerFilt}

    \subsubsection{Ferroelectricity}
        "The discovery of ferroelectricty in barium titanate in the 1940s made available for ceramic capacitor design dielectric constants up to 2 orders of magnitude greater than previously known."
        \cite[Ch 3 Sec III]{cerMaterials}

        Preexisting dipoles can be accurately predicted in ferroelectric materials' crystal structure. They create regions of uniform polarization called domains. The crystal symmetry of the dielectric determines the relative orientation between domains.

        These domains can be forced into a particular orientation by an externally applied electric field. "The effect of which is to increase the component of polarization in the field direction."

        Increasing DC bias represses domain reversibility and consequently decreases the low-signal ac dielectric constant.

        Increasing frequency decreases the dielectric constant through a 'relaxation' "centered in the GHz range." At this point the DF is maximal.
        \cite[Ch 3 Sec III]{cerMaterials}

    \subsubsection{MISC}
        \begin{enumerate}
            \item Barium Titanate ($BaTi0_3$) \cite[Ch 3 Sec IV.C]{cerMaterials}
            \item Most modern ceramic capacitor dielectric. In common use now.
            \item First published on in the US in 1942.
            \item Took over a decade before the industry put the material into active development.
            \item It is usually blended with other materials to get the desired effect.
        \item 1941 -- barium titanate became available (eps = 1000) \cite{deis_hist}
        \end{enumerate}

\subsection{Paper Capacitors}
    \subsubsection{Misc}
        Paper capacitors use waxed paper as a dielectric. They are primarily used in high voltage applications. But they are not preferred for much due to their high leakage and tolerances.\cite{learn_caps}

        1876 -- wax impregnated paper dielectric with foil electrodes \cite{deis_hist}
        -> power supply filtering in radios

        WWI spurred the development of radio. "It might be considered that this period (the early 1920s) saw the birth of the components industry.... Paper-dielectric capacitors were mainly tubular types enclosed in plain bakalizied cardboard tubes, with bitman or similar material sealing the ends." \cite[ch3]{dumInv}

        Paper capacitors were first patented in 1876 by Fitzgerald. \cite[ch.~11]{dumInv}

        During WW2, "waxed tubular paper-dielectric capacitors were replaced by metal-cased tubular types with rubber end seals. Metalized paper-dielectric capacitors were developed. \cite[ch.~3]{dumInv}

        Kraft paper capacitors typically had a dielectric constant of about 2-6 \cite[ch.~8.1]{poorIntro}

    \subsubsection{Impregnated Paper Capacitor}
        Paper is used as the dielectric. The paper is soaked with mineral oil to provide extra stability. The paper is interleaved with metal foil and then rolled to make the capacitor. \cite[ch.~8.2.1.1]{poorIntro}

        "
        \begin{table}
        \begin{tabular}{| l | l | }
        \hline
            Capacitor Range          & 1nF to 1uF            \\ \hline
            Capacitor Tolerance      & +/-15\%               \\ \hline
            Maximum Voltage Ratings  & upto 2kV              \\ \hline
            Power Factor             & 0.005 to 0.01 (@1kHz) \\ \hline
            Temperature co-efficient & 100 to 200 ppm/$^oC$  \\ \hline
        \end{tabular}
        \caption{Impregnated Paper Capacitor Characteristics}\cite[ch.~8.2.1.1]{poorIntro}
        \end{table}
        "
    
    \subsubsection{Metalized Paper Capacitor}
        "The metalized paper capacitor is very similar to the impregnated paper capacitor. The exception is that instead of using a metal foil, one side of the paper dielectric is sprayed with metal." \cite{hist_cerFilt}
     "
        \begin{table}
        \begin{tabular}{| l | l | }
        \hline
            Capacitor Range          & 1nF to 1uF           \\ \hline
            Capacitor Tolerance      & +/-25\%              \\ \hline
            Maximum Voltage Ratings  & upto 1.5kV           \\ \hline
            Power Factor             & 0.02 (@1kHz)         \\ \hline
            Temperature co-efficient & 150 to 200 ppm/$^oC$ \\ \hline
            Temperature range        & $-40^oC$ to $80^oC$  \\ \hline
        \end{tabular}
        \caption{Metalized Paper Capacitor Characteristics}\cite[ch.~8.2.1.1]{poorIntro}
        \end{table}
     "

\subsection{Lead Zirconate Capacitors}
    The discover of Lead Zirconate came directly from the study of Barium-Titanate.

\subsection{Tantalum Capacitors}
Tantalum capacitors are typically electrolytic and mostly have much better characteristics than aluminum electrolytic caps. There downside is a lower "maximum capacitance and and lower maximum working voltage."\cite{learn_caps}

    1956 - Tantalum solid electrolytic capacitor
    Transistors - spurred need for solid electrolytic capacitor (coupling or bypass) \cite[f.~56-64]{dumInv}

    Tantalum capacitors began to be commercially produced in the 1950s with the onset of the transistor. Bell labs patented the first solid-tantalum capacitor with Sprague pantenting the first comercially vaiable design in 1960. The solid tantalum capacitor presented a component which had a much higher capacitance per unit volume and greater reliablity than the market offerings at that time. \cite{charTant}

    Sprague released the first surface mount tantalum capacitor in the 1970s.\cite{spragueHist}

    Tantalum was designed to be a replacement for alumunim electrolytic capcitors in certain circumstances. It had a greater dielectric constant and was much more stable than aluminum. And with solid tantalum capacitors, the detriments of a liquid electrolyte went away. One of the historical main draw backs with tantalum capacitors has been the volatility in the supply chain. Short price spikes have repeatedly occurred in the market. When one such spike occurred around 1980, finer grain tantalum powders were created to acheive greater capacitance, in a smaller package with less overall tantalum. \cite[ch~3.1]{tantMis}

\subsection{Titanium Capacitors}

\subsection{Old Intro}
    \begin{enumerate}
        \item glass (porcelein)
        \item Steatite ceramics

        Contained feldspar and were not as high loss as porcelein ceramics
        \item paper and mica 
        
        Were primarily used in first radio circuits on the early 1900s.
        They had very good loss characteristics, and could be manufactured into units with thin sheets (high capacitance, despite having a modest dielectric constant).
        \item Germany 1920 -- steatite ceramics that used alkaline earth oxide fluxes instead of alkaline oxide (feldspare)

        This gave them a low-loss characteristic.
        \item Extruded tubular steatite ceramic capacitors arrived and replaced mica
    \end{enumerate}
    \cite[Ch 3 Sec II]{cerMaterials}



\subsection{General Classifications}
As of 1998: there are 5 basic capacitor technologies:
"
\begin{enumerate}
    \item ceramic
    \item aluminum electrolytics
    \item tantalum electrolytics
    \item film (polymeric)
    \item film (mica and paper)
\end{enumerate}
"
\cite{capRev}

3 basic technologies:
"
\begin{enumerate}
    \item electrolytic
    \item film
    \item ceramic
\end{enumerate}
"
\cite{capRev}

distinctions
"

aluminum - liquid impregnant
tantalum - dry impregnant

polymeric and mica -- synthetic film
paper film         -- natural film
"
\cite{capRev}

Evolving technologies:

CDL (Chemical double layer)
\begin{itemize}
    \item very high capacitances
    \item 1-3V working
    \item fuly bipolar
\end{itemize}

nanostructure mulitlayer
\begin{itemize}
    \item inorganic high dielectric constant coatings
    \item interleaved layers
    \item 100-10,000 layers
\end{itemize}
\cite{capRev}

\subsection{Electrolytic Capactors}

Characteristics:
\begin{itemize}
    \item moderate energy
    \item moderate power density
    \item relatively high losses
    \begin{itemize}
        \item high ESR
        \item high DF
    \end{itemize}
    \item polarity dependent
    \item primarily used in dc circuits
\end{itemize}

Typical Uses:
\begin{itemize}
    \item large value capacitors (1-100,000 uF)
    \item DC applications
    \begin{itemize}
        \item filtering
        \item rectified circuits
        \item pulsing circuits
        \begin{itemize}
            \item strobe lights
            \item SCR communication circuits
            \item fractional horsepower motor starting
        \end{itemize}
    \end{itemize}
\end{itemize}
\cite{capRev}

Karol Pollak discovered the principle of the electrolytic capacitor in 1886 while he was researching the anodization of metals. In 1897, he received a patent for the borax-solution aluminum electrolytic capacitor. In 1926, Julius Lilienfeld patented an electrolytic capacitor containing electrolyte soaked paper. Ralph D. Mershon developed the first practical, commercially available radio electrolytic capacitor. After the start of WW2, increased funding and effort was applied to the cause of electrolytic capacitors and technics such as "etching and pre-anodizing" greatly increased their reliability. \cite{wiki_elec}

In 1897, Charles Pollak received a patent for the borax electrolyte aluminum electrolytic capacitor. In 1936 Cornell-Dubilier opened a factory to produce aluminum electrolytic capacitors. EL caps became more reliable after WW2 after people threw resources at them. \cite{deis_hist}

An electrolyte is an "ionic conducting fluid." \cite{radio_elec}
Mclean and Power developed the tantalum solid electrolytic capacitor in 1956. \cite{radio_elec}

\subsection{Cornell-Dublier}
William Dublier invented the first pratical mica capacitor in 1909. He went on to found the Dublier Condensor Company in 1915, which became the modern Cornell-budilier in 1933.\cite{deis_hist} 

\subsection{Capacitor Markets}
\begin{itemize}
    \item motors
    \item lighting
    \item power supplies
    \item electronic circuits
\end{itemize}
\cite{capRev}

\subsection{Mica Capacitors}
During WWI, mica capacitors began to be produced in large quantities. This was mainly due to it being able to survive shock from weapons better than its glass counterpart. Also, it allowed the capacitors to be shrunk to achieve the same purpose. \cite[f.~37-41]{dumInv}

In light of mica supply chain problems and the emergence of ceramic capacitors during WW2, mica capacitors fell from prominence to a niche market after the second great war.\cite[Ch 3, Sec II]{cerMaterials}

Mica capacitors were originally made as a "clamped" style capacitor and used through the 1920s\cite{wiki_mica}. 

The mica capacitor was invented by M. Bauer in Germany in 1874. 

Mica capacitors are used as an extremely reliable and efficient option due to mica's inherent inertness and stability. \cite{tedds_mica}    
--
"Mica is very stable electrically, mechanically, and chemically. $\epsilon$ ~= [5-7].\cite{uiowa_mica}"

Mica capacitors have a very low tempco, change little with voltage, high Q, low ELS, and vary little with frequency. Used in RF a lot. On the downside, they are large. (You can used ceramics as a replacement). Most mica capacitors have low values (in the nF range).\cite{uiowa_mica}

\cite{hh2}[ch 1.13 pg 22]
\nocite{capGuide_mica}

Mica capacitors tend to provide very good results. They provide a choice which has high accuracy, a low temco, low variations of capacitance with voltage, and high Q. There main drawback is that they tend to be available only in lower capacitance ranges. They are know to jump in value on some occasions???? 
\cite{radio_mica}

The mica dielectric "was also one of the first dielectric materials to be used for capacitors in the early days of wireless because of its combination of stability and general physical and mechanical stability." Mica capacitors age slowly due to mica's natural inertness (it reacts with very little).\cite{radio_mica}

\subsubsection{Silver Mica Capacitors}

Tight tolerance (<=1\%),really good tempco and stability, but low capacitance values and expensive. Used in high frequency circuits a lot.\cite{learn_caps}

\subsection{Aluminum Capacitors}

Aluminum capacitors typically come in up to 50mF, but have high leakage currents, large tolerances, and large temperature coefficients. \cite{learn_caps}

\subsection{Polyester Capacitors}

Polyester capacitors provide an option for a cheap, high capacitance per unit volume option. On the downside, they have a high temperature coefficient. \cite{learn_caps}

\subsection{Polypropylne Capacitors}

Polypropylene capacitors can come up to a high working voltage of up to at least 3kV. They also can be used as a replacement to polyester capacitors due to their tight (1\%) tolerances.\cite{learn_caps}

\subsection{Polystyrene Capacitors}

Polystyrene capacitors have a high isolation resistance, but they also have a high ESL and exhibit a permanent value change if they get above 70C.
\cite{learn_caps}

\subsection{Polycarbonate Capcitors}

Polycarbonate capcitors have descent temperature coeficients, but bad tolerance.  \cite{learn_caps}

\subsection{Film}

"Traditional film capacitors were only available in modest sizes, <10uF."

Longer lifespans when compared to electrolytics.

Types:
\begin{itemize}
    \item Film-foil -- Good at high currents
    \item metalized filem -- Good at self-healing
\end{itemize}

"

Pros:
\begin{enumerate}
    \item low leakage
    \item low aging
\end{enumerate}

Cons
\begin{enumerate}
    \item low K
\end{enumerate}

"
\cite{capSite_intro}

\subsection{Applications}

In 1907, Lee De Forest created the first? vaccum tube. \cite{deis_hist}
1915 -- coast to coast telephone
1927 -- RCA Radiola 17 -- first AC line powered radio

mica dielectrics -- RF circuits

1909 -- William Dublier -- invented mica dielectric capacitors
--> highly reliable

1897 -- Charles Pollak got the patent for the first electrolytic capcitor (borax electrolyte aluminum)

WW2 -- electrolytic capacitors became more reliable

\subsection{Aluminum Capacitors}

Aluminum capacitors typically come in up to 50mF, but have high leakage currents, large tolerances, and large temperature coefficients. \cite{learn_caps}

\subsection{Polyester Capacitors}

Polyester capacitors provide an option for a cheap, high capacitance per unit volume option. On the downside, they have a high temperature coefficient. \cite{learn_caps}

\subsection{Polypropylne Capacitors}

Polypropylene capacitors can come up to a high working voltage of up to at least 3kV. They also can be used as a replacement to polyester capacitors due to their tight (1\%) tolerances.\cite{learn_caps}

\subsection{Polystyrene Capacitors}

Polystyrene capacitors have a high isolation resistance, but they also have a high ESL and exhibit a permanent value change if they get above 70C.
\cite{learn_caps}

\subsection{Polycarbonate Capcitors}

Polycarbonate capcitors have descent temperature coeficients, but bad tolerance.  \cite{learn_caps}

\subsection{Applications}

In 1907, Lee De Forest created the first? vaccum tube. \cite{deis_hist}
1915 -- coast to coast telephone
1927 -- RCA Radiola 17 -- first AC line powered radio

mica dielectrics -- RF circuits


1909 -- William Dublier -- invented mica dielectric capacitors


--> highly reliable


1897 -- Charles Pollak got the patent for the first electrolytic capcitor (borax electrolyte aluminum)

WW2 -- electrolytic capacitors became more reliable


Metalized, self-clearing capacitors
-Mansbridge patented in 1900 -- frequent shorts
-WW2 - reached maturity
1950 - used in the phone system

1954 - Bell labs took the metalied... and created the metalized polymer film capacitor

1941 - Polyethlyne  tetrephtalate (PET) -- Mylar -- 1950s

1951 -  semicrystalline polypropelyne 

1970 - film-foil capacitors (no paper) - electric utility applications

1957 - Electric Double Layer Capcitors patented
--1978  - "supercapacitor" 5.5V 1F \cite{deis_hist}

\subsection{Current State of the Art}

Current Important Types: \cite{deis_hist}
\begin{enumerate}
    \item impregnated foil-polymer film (high V,I)
    \item metalized film
    \item ceramic
    \item electrolytic
    \item electric double layer
\end{enumerate}

\subsection{Strengths and Weaknesses}
Look here for more info. \cite{deis_hist}
