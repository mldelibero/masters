\section {Voltage Characteristics}

\noindent This section will contain notes on how capacitor's change with respect to a change in AC/DC voltage.

\noindent Capacitance changes with both AC and DC voltage. Typically, capcitors with large dielectric values have an increased change in capcitance due to a DC bias.\cite{murVolChar}

\noindent Large DC Effect:
\begin {enumerate}
    \item "High dielectric constant-type monolititc ceramic capacitors that use barium titanait-based ferroelectrics"
\end {enumerate}

\noindent Small DC Effect:
\begin {enumerate}
    \item Polymer Al
    \item Conductive polymer tantalum electrolytic caps (Polymer Ta)
    \item Film capacitors
    \item temperature-compensating-type monalithic ceramic capacitors (titanium oxid or calcium zirconate) (COG)
\end {enumerate}

\noindent As for capcitance change in response to a varying AC waveform, the result is generally less, and different. The capcitance tends to increase with increasing AC voltage. But with some MLCC (Y5V), it will start to decrease after a point.  

\noindent New article: \cite{mur_mlcc_cirDes}
\noindent Old standard for measuring capacitance was the \textless JIS C 5101-22\textgreater.
\noindent 0V/0.5V-1 Vrms
\noindent New standard for measuring capacitance is the \textless JEITA RCX-2326\textgreater.
\noindent 1.25/0.1 Vrms

\noindent The new standard is taking DC bias into account when measuring capacitance.

\cite{oracle}
\noindent "High AC bias increases capacitance at low DC bias."
\noindent "High AC bias decreases capacitance at hi DC bias."

\noindent The voltage dependence is noticed mostly in Class II MLCC's, but not in Class I. It has something to do with the materials used in each.
